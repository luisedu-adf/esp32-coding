\documentclass[a4paper,10pt]{article}
\usepackage[utf8]{inputenc}
\usepackage{geometry}
\usepackage{enumitem}
\usepackage{hyperref}

% Ajuste das margens
\geometry{margin=1in}

% Comandos customizados
\newcommand{\heading}[1]{\noindent\textbf{\Large{#1}}}
\newcommand{\subheading}[1]{\noindent\textbf{\normalsize{#1}}} % Sem espaçamento extra

\begin{document}

\begin{center}
    {\huge \textbf{Luis Eduardo Alencar Duarte Franco}} \\
    Teresina, Piauí, Brasil \\
    \href{mailto:adf.luis@yahoo.com}{adf.luis@yahoo.com} \textbullet{} (86) 99986-2736 \\
    \href{https://github.com/seu-usuario}{GitHub} \textbullet{} \href{https://linkedin.com/in/seu-usuario}{LinkedIn}
\end{center}

\vspace{0.5cm}

\heading{Estudante de Ciência da Computação} \\ 
Animado para resolver problemas reais, com experiência prática em métodos ágeis, configuração de ambiente, Linux, CI/CD e mais. Proativo, comunicativo e sempre disposto a aprender, com dedicação e disposição para novos desafios.

\vspace{0.5cm}

\heading{Educação} \\
\textbf{Universidade Federal do Piauí} — Teresina, Piauí, Brasil \\
Bacharelado em Ciência da Computação (Jan 2019 – Dez 2026)

\vspace{0.5cm}

\heading{Projetos}

\vspace{0.3cm}

\noindent
\subheading{\href{https://github.com/luisedu-adf/risc16}{RiSC-16 Processor Implementation}}  
\begin{itemize}[noitemsep]
    \item Desenvolvido usando VHDL, GHDL e GTKWave, simulando instruções completas do conjunto RiSC-16.
    \item Automatizou testes de desempenho utilizando Bash Scripts.
\end{itemize}


\noindent
\subheading{\href{https://github.com/es20231/eqp3}{Web App - CRUD para Imagens}}  
\begin{itemize}[noitemsep]
    \item API RESTful (Flask) integrada com front-end React.js.
    \item Utilizou metodologias ágeis como GitActions e MVC em trabalho colaborativo.
\end{itemize}


\noindent
\subheading{\href{https://github.com/luisedu-adf/chycho}{Web Scraper e Blog Redesign}}  
\begin{itemize}[noitemsep]
    \item Reconstrução de um blog com BeautifulSoup e MongoDB, realizando extração de dados por web scraping.
    \item Implantado em uma instância AWS EC2, mantendo o serviço online por 1 ano.
\end{itemize}

\vspace{0.5cm}

\heading{Habilidades Técnicas}
\begin{itemize}[noitemsep]
    \item \textbf{Linguagens:} Python, Java, C, VHDL, HTML, CSS
    \item \textbf{Frameworks:} Flask, React.js
    \item \textbf{Testes:} CI/CD, Pytest, Selenium
    \item \textbf{Banco de Dados:} MySQL, SQLite, MongoDB
    \item \textbf{Ferramentas:} Git, Postman, LaTeX, PlantUML
    \item \textbf{Idiomas:} Inglês Fluente
\end{itemize}

\vspace{0.5cm}



\heading{Certificações}
\begin{itemize}[noitemsep]
    \item \href{https://www.credly.com/badges/fe74fb6a-74be-44ed-be9a-71f450ec811a/linked_in_profile}{AWS Cloud Practitioner Certification}
\end{itemize}

\end{document}

